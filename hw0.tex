\documentclass[11pt]{scrartcl}
\usepackage{dominatrix}

\title{Homework 0}
\subject{Statistical Machine Learning (STAT W4400)}
\author{Linan Qiu\\\texttt{lq2137}}
\begin{document}
\maketitle

Let 

\[
A = \begin{bmatrix} 1 & 2 \\ 2 & 4 \end{bmatrix}, \; B = \begin{bmatrix} 1 & 2 \\ 3 & 4 \end{bmatrix}, \; x = \begin{bmatrix} 2 \\ 1 \end{bmatrix}
\]

\begin{enumerate}
\item $B_{2,1}$ is $3$

\item $A+B = \begin{bmatrix} 2 & 4 \\ 5 & 8 \end{bmatrix}$

\item $AB = \begin{bmatrix} 1 * 1 + 2 * 3 & 1 * 2 + 2 * 4 \\ 2 * 1 + 4 * 3 & 2 * 2 + 4 * 4 \end{bmatrix} = \begin{bmatrix} 7 & 10 \\ 14 & 20 \end{bmatrix}$

\item $rank(A) = 1$

\begin{align*}
Av &= \lambda v \\
(A - \lambda I)v &= 0 \\
det(A - \lambda I) &= 0 \\
&= (1-\lambda)(4-\lambda)-4 \\
&= \lambda^2 - 5\lambda \\
&= \lambda(\lambda-5) 
\end{align*}
\item Largest eigenvalue of $A$ is $5$

\begin{align*}
(A - \lambda I) v &= 0 \\
(A - 5I) v &= 0 \\
\begin{bmatrix} 1 - 5 & 2 \\ 2 & 4 - 5 \end{bmatrix} \begin{bmatrix} v_1 \\ v_2 \end{bmatrix} &= 0
\end{align*}
Then,
\begin{align*}
-4v_1 + 2v_2 &= 0 \\
2v_1 -1v_2 &= 0
\end{align*}
Then, let $v_2 = t$, $v_1 = -2t$. Then, the eigenspace corresponding to $\lambda = 5$ is given by the span of $\begin{bmatrix} -2 \\ 1 \end{bmatrix}$.
\item Eigenvector associated is the span of $v = \begin{bmatrix} 0 \\ 0 \end{bmatrix}$

\item $|B| = 1*4 - 2*3 = -2$

\item $x^T Ax = \begin{bmatrix} 2 & 1 \end{bmatrix} \begin{bmatrix} 1 & 2 \\ 2 & 4 \end{bmatrix} \begin{bmatrix} 2 \\ 1 \end{bmatrix} = 16$

\item $x^T x = \begin{bmatrix} 2 & 1 \end{bmatrix} \begin{bmatrix} 2 \\ 1 \end{bmatrix} = 5$

\item $x x^T = \begin{bmatrix} 2 \\ 1 \end{bmatrix} \begin{bmatrix} 2 & 1 \end{bmatrix} = \begin{bmatrix} 4 & 2 \\ 2 & 1 \end{bmatrix}$

\item $||x||_2 = \sqrt{(2-1)^2} = 1$

\item 

\item

\item $n=2$

\begin{align*}
Var(y) &= E(y)^2 - E(y^2) \\
E(y^2) &= E(y)^2 - Var(y) \\
&= \mu^2 - \sigma^2
\end{align*}
\item $E(y^2) = \mu^2 - \sigma^2$

\item $y + w \sim N(2.7 + 3.1, \sqrt{8^2 + 15^2}) = N(5.8, 17)$

\item The normalizing constant is $(2\pi)^{\frac{p}{2}} |\Sigma|^{\frac{1}{2}}$ where $p=2$ and $|\Sigma|$ is the determinant of the $\Sigma$ given.

\item The support of a Bernoulli random variable is $\{0,1\}$

\item Define the Bernoulli process as $N(k,n)$. Then $N(k,n) = \begin{pmatrix}n \\ k\end{pmatrix} = \frac{n!}{k!(n-k)!}$

\begin{align*}
h(x_1) &= \frac{1}{3}x_1^3 - \frac{1}{2}x_1^2 - 6x_1 + \frac{27}{2} \\
h'(x_1) &= x_1^2 - x_1 - 6 \\
h''(x_1) &= 2x_1 - 1
\end{align*}
Global minima/maxima is at
\begin{align*}
h'(x_1) &= x_1^2 - x_1 - 6 = 0 \\
(x_1 - 3)(x_1+2) &= 0
\end{align*}
$x_1 = 3$ is a minima since $h''(3) > 0$. $x_1 = -2$ is a global maxima since $h''(-2) < 0$
\item $x_1 = -2$

\item $x_1 = 3$

\item 

\item

\item

\item

\end{enumerate}



\end{document}
